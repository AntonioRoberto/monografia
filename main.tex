%%%%%%%%%%%%%%%%%%%%%%%%%%%%%%%%%%%%%%%%%
% Masters/Doctoral Thesis 
% LaTeX Template
% Version 2.1 (2/9/15)
%
% This template has been downloaded from:
% http://www.LaTeXTemplates.com
%
% Version 2.0 major modifications by:
% Vel (vel@latextemplates.com)
%
% Original authors:
% Steven Gunn  (http://users.ecs.soton.ac.uk/srg/softwaretools/document/templates/)
% Sunil Patel (http://www.sunilpatel.co.uk/thesis-template/)
%
% License:
% CC BY-NC-SA 3.0 (http://creativecommons.org/licenses/by-nc-sa/3.0/)
%
%%%%%%%%%%%%%%%%%%%%%%%%%%%%%%%%%%%%%%%%%

%----------------------------------------------------------------------------------------
%	PACKAGES AND OTHER DOCUMENT CONFIGURATIONS
%----------------------------------------------------------------------------------------

\documentclass[
11pt, % The default document font size, options: 10pt, 11pt, 12pt
%oneside, % Two side (alternating margins) for binding by default, uncomment to switch to one side
brazilian, % ngerman for German
singlespacing, % Single line spacing, alternatives: onehalfspacing or doublespacing
%draft, % Uncomment to enable draft mode (no pictures, no links, overfull hboxes indicated)
%nolistspacing, % If the document is onehalfspacing or doublespacing, uncomment this to set spacing in lists to single
%liststotoc, % Uncomment to add the list of figures/tables/etc to the table of contents
%toctotoc, % Uncomment to add the main table of contents to the table of contents
%parskip, % Uncomment to add space between paragraphs
]{MastersDoctoralThesis} % The class file specifying the document structure

\usepackage[utf8]{inputenc} % Required for inputting international characters
\usepackage[T1]{fontenc} % Output font encoding for international characters

\usepackage{palatino} % Use the Palatino font by default

\usepackage[backend=bibtex,style=authoryear,natbib=true]{biblatex} % User the bibtex backend with the authoryear citation style (which resembles APA)

\addbibresource{example.bib} % The filename of the bibliography

\usepackage[autostyle=true]{csquotes} % Required to generate language-dependent quotes in the bibliography
\usepackage{amssymb}
\usepackage{amsmath}
\usepackage{hyperref}
\usepackage{algorithm}
\usepackage[noend]{algpseudocode}
\usepackage{hyperref}
%\usepackage[]{algorithm2e}
%\usepackage{algorithmic}
%\usepackage[colorlinks=true,linkcolor=blue]{hyperref} 

\newtheorem{theorem}{Teorema}
\newtheorem{definition}{Definição}
\newtheorem{proof}{Demonstracão}
\newtheorem{corollary}{Corolário}

\makeatletter
\def\BState{\State\hskip-\ALG@thistlm}
\makeatother


%----------------------------------------------------------------------------------------
%	THESIS INFORMATION
%----------------------------------------------------------------------------------------

\thesistitle{Teoria dos Números e Computação: Uma abordagem utilizando problemas de competições de programação} % Your thesis title, this is used in the title and abstract, print it elsewhere with \ttitle
\supervisor{Dr. Carlos Eduardo Ferreira} % Your supervisor's name, this is used in the title page, print it elsewhere with \supname
\examiner{} % Your examiner's name, this is not currently used anywhere in the template, print it elsewhere with \examname
\degree{Bacharel em Ciência da Computação} % Your degree name, this is used in the title page and abstract, print it elsewhere with \degreename
\author{Antonio R. de Campos Junior} % Your name, this is used in the title page and abstract, print it elsewhere with \authorname
\addresses{} % Your address, this is not currently used anywhere in the template, print it elsewhere with \addressname

\subject{Ciência da Computação} % Your subject area, this is not currently used anywhere in the template, print it elsewhere with \subjectname
\keywords{} % Keywords for your thesis, this is not currently used anywhere in the template, print it elsewhere with \keywordnames
\university{\href{http://www5.usp.br/}{Universidade de São Paulo}} % Your university's name and URL, this is used in the title page and abstract, print it elsewhere with \univname
\department{\href{http://www.ime.usp.br/}{Instituto de Matemática e Estatística}} % Your department's name and URL, this is used in the title page and abstract, print it elsewhere with \deptname
\group{\href{http://researchgroup.university.com}{Research Group Name}} % Your research group's name and URL, this is used in the title page, print it elsewhere with \groupname
\faculty{\href{http://faculty.university.com}{Faculty Name}} % Your faculty's name and URL, this is used in the title page and abstract, print it elsewhere with \facname

\hypersetup{pdftitle=\ttitle} % Set the PDF's title to your title
\hypersetup{pdfauthor=\authorname} % Set the PDF's author to your name
\hypersetup{pdfkeywords=\keywordnames} % Set the PDF's keywords to your keywords

\begin{document}

\frontmatter % Use roman page numbering style (i, ii, iii, iv...) for the pre-content pages

\pagestyle{plain} % Default to the plain heading style until the thesis style is called for the body content

%----------------------------------------------------------------------------------------
%	TITLE PAGE
%----------------------------------------------------------------------------------------

\begin{titlepage}
\begin{center}

\textsc{\LARGE \univname}\\[1.5cm] % University name
\textsc{\Large Trabalho de Formatura}\\[0.5cm] % Thesis type

\HRule \\[0.4cm] % Horizontal line
{\huge \bfseries \ttitle}\\[0.4cm] % Thesis title
\HRule \\[1.5cm] % Horizontal line
 
\begin{minipage}{0.4\textwidth}
\begin{flushleft} \large
\emph{Autor:}\\
\href{http://www.ime.usp.br/~arcjr}{\authorname} % Author name - remove the \href bracket to remove the link
\end{flushleft}
\end{minipage}
\begin{minipage}{0.4\textwidth}
\begin{flushright} \large
\emph{Supervisor:} \\
\href{http://www.ime.usp.br/~cef/}{\supname} % Supervisor name - remove the \href bracket to remove the link  
\end{flushright}
\end{minipage}\\[3cm]
\leavevmode \\
\leavevmode \\
\leavevmode \\
\leavevmode \\

\large \textit{Tese apresentada em cumprimento dos requisitos para o curso \\ \degreename}\\[0.3cm] % University requirement text
%\textit{in the}\\[0.4cm]
\deptname\\[2cm] % Research group name and department name
 
{\large \today}\\[4cm] % Date
%\includegraphics{Logo} % University/department logo - uncomment to place it
 
\vfill
\end{center}
\end{titlepage}

%----------------------------------------------------------------------------------------
%	DECLARATION PAGE
%----------------------------------------------------------------------------------------

%----------------------------------------------------------------------------------------
%	QUOTATION PAGE
%----------------------------------------------------------------------------------------

\vspace*{0.2\textheight}

\noindent\enquote{\itshape To raise new questions, new possibilities, to regard old problems from a new angle, requires creative imagination and marks real advance in science.}\bigbreak

\hfill Albert Einstein
\newpage
%----------------------------------------------------------------------------------------
%	ABSTRACT PAGE
%----------------------------------------------------------------------------------------

\chapter*{Resumo}
\thispagestyle{empty}

Teoria do Números é um vasto ramo da matemática que estuda números inteiros. Números primos, fatorização de números inteiros, funções aritméticas, são alguns dos tópicos mais estudados e também importantes para resolução de problemas computacionais.

Hoje em dia a importância da Teoria do Números na Computação é inquestionável, e desse modo, esse trabalho vem ilustrar como a teoria pode ser aplicada na criação de algoritmos para resolução de problemas computacionais, em especial problemas de competições de programação.

Equações diofantinas, Congruência Modular, Números de Fibonacci, são alguns dos assuntos que serão abordados nesse trabalho. Após a devida demostração da teoria serão exibidos alguns problemas de competições de programação que aplicam essa teoria, seguido da implementação e análise do algoritmo que resolve o problema abordado.

\clearpage

%\section{abstract}
%\addchaptertocentry{\abstractname} % Add the abstract to the table of contents

%The Thesis Abstract is written here (and usually kept to just this page). The page is kept centered vertically so can expand into the blank space above the title too\ldots

%\end{abstract}

%----------------------------------------------------------------------------------------
%	ACKNOWLEDGEMENTS
%----------------------------------------------------------------------------------------

\chapter*{Agradecimentos}
\thispagestyle{empty}

I like to acknowledge ...

\clearpage
%\section{acknowledgements}
%\addchaptertocentry{\acknowledgementname} % Add the acknowledgements to the table of contents

%The acknowledgements and the people to thank go here, don't forget to include your project advisor\ldots

%\end{acknowledgements}

%----------------------------------------------------------------------------------------
%	LIST OF CONTENTS/FIGURES/TABLES PAGES
%----------------------------------------------------------------------------------------

\tableofcontents % Prints the main table of contents

\listoffigures % Prints the list of figures

\listoftables % Prints the list of tables

%----------------------------------------------------------------------------------------
%	ABBREVIATIONS
%----------------------------------------------------------------------------------------

%\section{abbreviations}{ll} % Include a list of abbreviations (a table of two columns)

%\textbf{LAH} & \textbf{L}ist \textbf{A}bbreviations \textbf{H}ere\\
%\textbf{WSF} & \textbf{W}hat (it) \textbf{S}tands \textbf{F}or\\

%\end{abbreviations}

%----------------------------------------------------------------------------------------
%	PHYSICAL CONSTANTS/OTHER DEFINITIONS
%----------------------------------------------------------------------------------------

%\section{constants}{lr@{${}={}$}l} % The list of physical constants is a three column table

% The \SI{}{} command is provided by the siunitx package, see its documentation for instructions on how to use it

%Speed of Light & $c$ & \SI{2.99792458e8}{\meter\per\second} (exact)\\
%Constant Name & $Symbol$ & $Constant Value$ with units\\

%\end{constants}

%----------------------------------------------------------------------------------------
%	SYMBOLS
%----------------------------------------------------------------------------------------

%\section{symbols}{lll} % Include a list of Symbols (a three column table)

%$a$ & distance & \si{\meter} \\
%$P$ & power & \si{\watt} (\si{\joule\per\second}) \\
%Symbol & Name & Unit \\

%\addlinespace % Gap to separate the Roman symbols from the Greek

%$\omega$ & angular frequency & \si{\radian} \\

%\end{symbols}

%----------------------------------------------------------------------------------------
%	DEDICATION
%----------------------------------------------------------------------------------------

\dedicatory{For/Dedicated to/To my\ldots} 

%----------------------------------------------------------------------------------------
%	THESIS CONTENT - CHAPTERS
%----------------------------------------------------------------------------------------

\mainmatter % Begin numeric (1,2,3...) page numbering

\pagestyle{thesis} % Return the page headers back to the "thesis" style

% Include the chapters of the thesis as separate files from the Chapters folder
% Uncomment the lines as you write the chapters

% Chapter 1

\chapter{Introdução} % Main chapter title

\label{Chapter1} % Change X to a consecutive number; for referencing this chapter elsewhere, use \ref{ChapterX}

%----------------------------------------------------------------------------------------
%	SECTION 1
%----------------------------------------------------------------------------------------

Nesse trabalho são apresentados vários tópicos relacionados à Teoria dos Números e técnicas de programação.
\newline

Todos os capítulos são divididos em duas seções principais:

\begin{itemize}

\item A primeira seção mostra alguns resultados famosos na área de Teoria dos Números. Todas as proposições e teoremas abordados são devidamente demostrados.

\item A segunda seção ("Problemas Propostos") propõe alguns problemas de Competições de Programação (retirados em sua maioria do site \url{http://a2oj.com/}), com respectivo
pseudocódigo e análise de complexidade.

\end{itemize}

Durante a leitura, é possível testar suas próprias soluções com algum dos \textbf{Online Judges} (plataformas de correção online) citados no \textbf{Apêndice B}.


% Chapter 2

\chapter{Congruência} % Main chapter title

\label{Chapter2} % Change X to a consecutive number; for referencing this chapter elsewhere, use \ref{ChapterX}

%----------------------------------------------------------------------------------------
%	SECTION
%----------------------------------------------------------------------------------------

\section{Congruência}



%----------------------------------------------------------------------------------------
%	SECTION
%----------------------------------------------------------------------------------------

\section{Congruência Linear}



%----------------------------------------------------------------------------------------
%	SECTION
%----------------------------------------------------------------------------------------

\section{Teorema de Fermat, Euler e Wilson}



%----------------------------------------------------------------------------------------
%	SECTION
%----------------------------------------------------------------------------------------

\section{Teorema do Resto Chinês}

\begin{theorem}[Teorema do Resto Chinês]
Tome o sistema de congruências lineares:\\

$a_1x \equiv c_1 (mod$ $m_1)$\\
$a_2x \equiv c_2 (mod$ $m_2)$\\
$a_3x \equiv c_3 (mod$ $m_3)$\\
$...$\\
$a_nx \equiv c_n (mod$ $m_n)$\\

Em que $c_i \in \mathbb{Z}$, $MDC(a_i,m_i) = 1$, e $MDC(m_i, m_j) = 1$ para $i \neq j$
Nessas condições o sistema acima tem solução única módulo $M$, em que $M = m_1m_2m_3...m_n$.
\end{theorem}
\textbf{Demonstração:}
Deixaremos a demostração a cargo do leitor.

%----------------------------------------------------------------------------------------
%	SECTION
%----------------------------------------------------------------------------------------

\section{Problemas Propostos}



%----------------------------------------------------------------------------------------
\subsection{UVA-10090}
\href{https://uva.onlinejudge.org/index.php?option=onlinejudge&page=show_problem&problem=1031}{10090 - Marbles}\\


\textbf{Resumo:}
É dado um número inteiro $n$ ($0 < n \leq 10^8$). O problema consite em verificar se $n$ pode, ou não pode, ser escrito como a soma de dois números primos.
E em caso afirmativo encontrar o valor desses dois primos.
\\

\textbf{Solução:}
\\

\textbf{Pseudocódigo:}
\begin{algorithm}
\caption{Marbles}
\begin{algorithmic}[1]
\Procedure{FindTwoPrimesSum (n)}{}

\EndProcedure
\end{algorithmic}
\end{algorithm}


\textbf{Análise:}


 
% Chapter 3

\chapter{Funções Aritméticas} % Main chapter title

\label{Chapter3} % Change X to a consecutive number; for referencing this chapter elsewhere, use \ref{ChapterX}

%----------------------------------------------------------------------------------------
%	SECTION
%----------------------------------------------------------------------------------------

\section{$\varphi$ de Euler}

\begin{definition}
A Função Totiente de Euler, denotada por $\varphi(n)$, é a função aritmética que conta o número 
de inteiros positivos menores ou iguais a $n$ que são primos entre si com $n$.

$\varphi(n) := |\{ x \in \mathbb{N}^{*} \mid MDC(x,n) = 1 \}|$
\end{definition}


\begin{theorem}\label{phi_potencia}
$\varphi(n^k) = n^{k-1}\varphi(n)$, para inteiros positiovos quaisquer $n$ e $k$. Em particular 
$\varphi(p^k) = (p^k - p^{k-1})$, para $p$ primo.%, e $k \in \mathbb{N}^{*}$.
\end{theorem}
\textbf{Demonstração:}



\begin{theorem}\label{phi_multiplicativa}
$\varphi(n)$ é função multiplicativa, ie, $\varphi(mn) = \varphi(m)\varphi(n)$ para $MDC(m,n) = 1$.
\end{theorem}
\textbf{Demonstração:}



\begin{theorem}[Fórmula Produto de Euler]
$\varphi(n) = n \prod_{p|n}(1 - \frac{1}{p})$
\end{theorem}
\textbf{Demonstração:}
Pelo Teorema X, \autoref{phi_potencia}, \autoref{phi_multiplicativa} segue as seguintes recorrências:

$\varphi(n) = \varphi(p_1^{a_1}p_2^{a_2}...p_k^{a_k})$

$\varphi(n) = \varphi(p_1^{a_1})\varphi(p_2^{a_2})...\varphi(p_k^{a_k})$

$\varphi(n) = (p_1^{a_1} - p_1^{a_1-1})(p_2^{a_2} - p_2^{a_2-1})...(p_k^{a_k} - p_k^{a_k-1})$

$\varphi(n) = p_1^{a_1}p_2^{a_2}...p_k^{a_k}(1 - 1/p_1)(1 - 1/p_2)...(1 - 1/p_k)$

$\varphi(n) = n \prod_{p|n}(1 - \frac{1}{p})$ $\square$

%----------------------------------------------------------------------------------------
%	SECTION
%----------------------------------------------------------------------------------------

\section{Sequência de Fibonacci}

\begin{definition}
\end{definition}


%----------------------------------------------------------------------------------------
%	SECTION
%----------------------------------------------------------------------------------------

\section{Problemas Propostos}


%----------------------------------------------------------------------------------------
\subsection{UVA-11424}
\href{https://uva.onlinejudge.org/index.php?option=onlinejudge&page=show_problem&problem=2419}{11424 - GCD - Extreme (I)} \\

\textbf{Resumo:}
É dado um inteiro positivo $N$ ($1 < N < 200001$). O problema consiste em calcular o mais rápido possível a expressão:

$G(N) = \sum_{i=1}^{N-1}\sum_{j=i+1}^{N}MDC(i,j)$.
\\

\textbf{Solução:}
Trivialmente a expressão acima pode ser calculada em tempo proporcional à $O(n^2log(N))$, porém essa solução consome muito tempo e não será aceita no Judge Online. Vamos então mostrar uma solução mais eficiente.
\\

Primeiramente reescrevemos a expressão acima da seguinte maneira:

$G(N) = \sum_{j=2}^N\sum_{i=1}^{j-1}MDC(i,j)$ ( $\rhd$ Observe que as expressão são equivalentes).

Tome agora a função $F(M) = \sum_{i=1}^{M-1}MDC(i, M)$ $\Rightarrow$ $G(N) = \sum_{j=2}^NF(j)$.

Sabemos que todos os valores resultantes do método $MDC(i,M)$ calculados em $F(M)$ são divisores de $M$. Desse modo, podemos reescrever $F(M)$ da seguinte maneira:

$F(M) = \sum_{i=1}^{M-1}MDC(i, M) = \sum_{l=1}^{n}\lambda_l d_l$, em que, $d_1, d_2,..., d_n$ são os divisores de $M$, $\lambda_l$ é o número de vezes que o divisor $d_l$ aparece na somatória $\sum_{i=1}^{M-1}MDC(i,N)$, e $n$ é o número de divisores de $M$.
\\

Pelo Corolario \autoref{divisibilidade_mdc} temos que: $MDC(i,M) = d_l \Rightarrow MDC(i/d_l,M/d_l) = 1$. Logo o número de vezes que o divisor $d_l$ aparece na somatória, será igual ao número de primos entre si com $(M/d_l)$, ie, $\lambda_l = \varphi(M/d_l)$.

Reescrevendo novamente $F(M)$, temos:

$F(M) = \sum_{i=1}^{M-1}MDC(i, M) = \sum_{l=1}^n \lambda_l d_l = \sum_{l=1}^n \varphi(M/d_l) d_l$.

$G(N) = \sum_{j=2}^N \sum_{l=1}^n \varphi(j/d_l)d_l$ $\square$.
\\

\textbf{Pseudocódigo:}
\begin{algorithm}
\caption{GCD - Etreme(I)}\label{gcd_extreme}
\begin{algorithmic}[1]
\Procedure{G (N)}{}
\State $\varphi[] \gets PHI(N)$
\State $solution \gets 0$
\For {$j$ := $2$ to $N$}
\For {\textbf{each} divisor $d$ de $j$}
\State $solution \gets solution + \varphi[j/d] d$
\EndFor
\EndFor
\State \Return{$solution$}
\EndProcedure
\end{algorithmic}
\end{algorithm}


\textbf{Análise:}
O método $PHI(N)$ na linha 2 consome tempo proporcional à $O(N\sqrt{N})$.

O número de divisores de $j$ é proporcional à $O(\sqrt{N})$, já que $j \leq N$.

Assim a complexidade das linhas 4, 5, 6 do algoritmo é $O(N\sqrt{N})$.

Complexidade final do algoritmo: $O(N\sqrt{N})$.

\textbf{OBS.:} Para resolver o problema no Judge Online será preciso armazenar as soluções usando \href{https://linux.ime.usp.br/~stefanot/mac499/template.pdf}{Programação Dinâmica}.






%----------------------------------------------------------------------------------------
\subsection{TJU-3506}
\href{http://acm.tju.edu.cn/toj/showp3506.html}{3506 - Euler Function} \\

\textbf{Resumo:}
São dados dois números positivos $n$, $m$ ($1 < n < 10^7$, $1 < m < 10^9$).
O problenas consiste em calcular a expressão: $\varphi(n^m) \bmod 201004.$
\\

\textbf{Solução:}
Pelo \autoref{phi_potencia}
\\

\textbf{Pseudocódigo:}
\begin{algorithm}
\caption{GCD - Etreme(I)}\label{gcd_extreme}
\begin{algorithmic}[1]
\Procedure{G (N)}{}
\State $\varphi[] \gets PHI(N)$
\State $solutuion \gets 0$
\For {$j$ := $2$ to $N$}
\For {\textbf{each} divisor $d$ de $j$}
\State $solution \gets solution + \varphi[j/d] d$
\EndFor
\EndFor
\State \Return{$solutuion$}
\EndProcedure
\end{algorithmic}
\end{algorithm}


\textbf{Análise:}


%----------------------------------------------------------------------------------------
\subsection{CodeChef-MODEFB}
\href{https://www.codechef.com/problems/MOREFB}{71544 - Another Fibonacci}\\

\textbf{Resumo:}
São dados dois números inteiros $N$, $K$ ($1 \leq N \leq 50000$, $1 \leq K \leq N$) e um conjunto $S \subset \mathbb{N}$ com $N$ elementos, tal que, $\forall s \in S, 1 \leq s \leq 10^9$.
O problenas consiste em calcular a expressão: $F(S) = \sum_{A \subset S \hspace{1mm} e\hspace{1mm} |A| = K}^{} Fib(sum(A))$, onde $sum(A) = \sum_{a \in A}a$. %$\varphi(n^m) \bmod 201004.$
\\

\textbf{Solução:}
\\

\textbf{Pseudocódigo:}
\begin{algorithm}
\caption{Another Fibonacci}
\begin{algorithmic}[1]
\Procedure{F (S)}{}

\EndProcedure
\end{algorithmic}
\end{algorithm}


\textbf{Análise:}




%----------------------------------------------------------------------------------------
\subsection{UVA-10311}
\href{https://uva.onlinejudge.org/index.php?option=onlinejudge&page=show_problem&problem=1252}{10311 - Goldbach and Euler}\\

\textbf{Resumo:}
É dado um número inteiro $n$ ($0 < n \leq 10^8$). O problema consite em verificar se $n$ pode, ou não pode, ser escrito como a soma de dois números primos. 
E em caso afirmativo encontrar o valor desses dois primos.
\\

\textbf{Solução:}
\\

\textbf{Pseudocódigo:}
\begin{algorithm}
\caption{Goldbach and Euler}
\begin{algorithmic}[1]
\Procedure{FindTwoPrimesSum (n)}{}

\EndProcedure
\end{algorithmic}
\end{algorithm}


\textbf{Análise:}



% Chapter Template

\chapter{Lista de Problemas} % Main chapter title

\label{Chapter4} % Change X to a consecutive number; for referencing this chapter elsewhere, use \ref{ChapterX}

%----------------------------------------------------------------------------------------
%	SECTION 1
%----------------------------------------------------------------------------------------

Esse capítulo contém uma lista de problemas que envolvem \textit{Teoria dos Números}
separados por nível de dificuldade. Os problemas aqui listados foram retirados do site
\href{http://ahmed-aly.com/Category.jsp?ID=41}{http://ahmed-aly.com/}.
\\

%----------------------------------------------------------------------------------------
%	1
%----------------------------------------------------------------------------------------
\section{Nível 1}
\begin{itemize}
\item \url{http://www.spoj.com/problems/NEG2/}
\item \url{http://www.spoj.com/problems/PON/}
\item \url{http://www.spoj.com/problems/TWOSQRS/}
\item \url{https://uva.onlinejudge.org/index.php?option=onlinejudge&page=show_problem&problem=347}
\item \url{http://acm.tju.edu.cn/toj/showp1868.html}
\item \url{https://uva.onlinejudge.org/index.php?option=onlinejudge&page=show_problem&problem=1176}
\item \url{https://uva.onlinejudge.org/index.php?option=onlinejudge&page=show_problem&problem=1889}
\item \url{https://uva.onlinejudge.org/index.php?option=onlinejudge&page=show_problem&problem=1080}
\item \url{https://uva.onlinejudge.org/index.php?option=onlinejudge&page=show_problem&problem=1614}
\item \url{http://www.spoj.com/problems/FACT0/}
\item \url{http://www.spoj.com/problems/CPRIME/}
\item \url{http://www.spoj.com/problems/FCTRL/}
\item \url{https://uva.onlinejudge.org/index.php?option=onlinejudge&page=show_problem&problem=1865}
\item \url{http://codeforces.com/problemset/problem/80/A}
\item \url{https://uva.onlinejudge.org/index.php?option=onlinejudge&page=show_problem&problem=1240}
\item \url{http://acm.sgu.ru/problem.php?contest=0&problem=106}
\item \url{http://www.codechef.com/problems/POWERMUL}
\item \url{http://www.spoj.com/problems/MARBLES/}
\item \url{http://www.spoj.com/problems/LCMSUM/}
\item \url{http://www.spoj.com/problems/ETF/}
\item \url{https://uva.onlinejudge.org/index.php?option=onlinejudge&page=show_problem&problem=627}
\item \url{http://codeforces.com/problemset/problem/122/A}
\item \url{https://uva.onlinejudge.org/index.php?option=onlinejudge&page=show_problem&problem=1335}
\item \url{https://uva.onlinejudge.org/index.php?option=onlinejudge&page=show_problem&problem=1068}
\item \url{http://codeforces.com/problemset/problem/114/A}
\item \url{https://uva.onlinejudge.org/index.php?option=onlinejudge&page=show_problem&problem=3565}
\item \url{https://uva.onlinejudge.org/index.php?option=onlinejudge&page=show_problem&problem=96}
\item \url{http://www.spoj.com/problems/LINES/}
\item \url{http://www.spoj.com/problems/PRIME1/}
\item \url{http://www.spoj.com/problems/DIV/}
\item \url{http://www.spoj.com/problems/DIVSUM/}
\item \url{http://codeforces.com/problemset/problem/230/B}
\item \url{https://uva.onlinejudge.org/index.php?option=onlinejudge&page=show_problem&problem=484}
\item \url{http://www.spoj.com/problems/TIPTOP/}
\item \url{http://www.spoj.com/problems/FACTCG2/}
\item \url{https://uva.onlinejudge.org/index.php?option=onlinejudge&page=show_problem&problem=400}
\item \url{http://codeforces.com/problemset/problem/158/D}
\end{itemize}


%----------------------------------------------------------------------------------------
%	2
%----------------------------------------------------------------------------------------
\section{Nível 2}

\begin{itemize}
\item \url{http://www.spoj.com/problems/FRACTION/}
\item \url{http://www.spoj.com/problems/DIV2/}
\item \url{http://www.spoj.com/problems/CRYPTO1/}
\item \url{http://www.spoj.com/problems/NDIVPHI/}
\item \url{http://codeforces.com/problemset/problem/235/A}
\item \url{http://www.spoj.com/problems/ODDDIV/}
\item \url{https://uva.onlinejudge.org/index.php?option=onlinejudge&page=show_problem&problem=1658}
\item \url{http://www.spoj.com/problems/TUTMRBL/}
\item \url{https://uva.onlinejudge.org/index.php?option=onlinejudge&page=show_problem&problem=855}
\item \url{https://uva.onlinejudge.org/index.php?option=onlinejudge&page=show_problem&problem=1431}
\item \url{http://www.spoj.com/problems/PAGAIN/}
\item \url{http://www.spoj.com/problems/PROOT/}
\item \url{http://www.spoj.com/problems/DIVSUM2/}
\item \url{http://www.spoj.com/problems/KPEQU/}
\item \url{http://codeforces.com/problemset/problem/199/A}
\item \url{http://codeforces.com/problemset/problem/236/B}
\item \url{http://www.spoj.com/problems/MINNUM/}
\item \url{http://codeforces.com/problemset/problem/26/A}
\item \url{http://www.spoj.com/problems/PSYCHON/}
\item \url{http://www.spoj.com/problems/GCDEX/}
\item \url{https://uva.onlinejudge.org/index.php?option=onlinejudge&page=show_problem&problem=1252}
\item \url{https://uva.onlinejudge.org/index.php?option=onlinejudge&page=show_problem&problem=1031}
\item \url{http://www.spoj.com/problems/DCEPCA03/}
\end{itemize}


%----------------------------------------------------------------------------------------
%	3
%----------------------------------------------------------------------------------------
\section{Nível 3}

\begin{itemize}
\item \url{http://www.spoj.com/problems/FACT1/}
\item \url{http://codeforces.com/problemset/problem/248/B}
\item \url{https://icpcarchive.ecs.baylor.edu/index.php?option=onlinejudge&page=show_problem&problem=3382}
\item \url{http://www.spoj.com/problems/MSE08H/}
\item \url{http://www.spoj.com/problems/SCRAPER/}
\item \url{http://www.spoj.com/problems/MSKYCODE/}
\item \url{http://codeforces.com/problemset/problem/150/A}
\item \url{http://codeforces.com/problemset/problem/59/B}
\item \url{http://codeforces.com/problemset/problem/221/B}
\item \url{http://www.spoj.com/problems/UCI2009B/}
\item \url{http://www.codechef.com/problems/FUNC}
\item \url{http://www.spoj.com/problems/HOMEW/}
\item \url{http://codeforces.com/problemset/problem/237/C}
\item \url{http://www.spoj.com/problems/SQFREE/}
\item \url{http://codeforces.com/problemset/problem/68/A}
\item \url{http://codeforces.com/problemset/problem/17/A}
\item \url{http://codeforces.com/problemset/problem/284/A}
\end{itemize}



%----------------------------------------------------------------------------------------
%	4
%----------------------------------------------------------------------------------------
\section{Nível 4}

\begin{itemize}
\item \url{}
\item \url{}
\end{itemize}



%----------------------------------------------------------------------------------------
%	5
%----------------------------------------------------------------------------------------
\section{Nível 5}

\begin{itemize}
\item \url{}
\item \url{}
\end{itemize}



%----------------------------------------------------------------------------------------
%	6
%----------------------------------------------------------------------------------------
\section{Nível 6}

\begin{itemize}
\item \url{}
\item \url{}
\end{itemize}



%----------------------------------------------------------------------------------------
%	7
%----------------------------------------------------------------------------------------
\section{Nível 7}

\begin{itemize}
\item \url{}
\item \url{}
\end{itemize}



%----------------------------------------------------------------------------------------
%	8
%----------------------------------------------------------------------------------------
\section{Nível 8}


\begin{itemize}
\item \url{}
\item \url{}
\end{itemize}


%----------------------------------------------------------------------------------------
%	9
%----------------------------------------------------------------------------------------
\section{Nível 9}

\begin{itemize}
\item \url{}
\item \url{}
\end{itemize}



%----------------------------------------------------------------------------------------
%	10
%----------------------------------------------------------------------------------------
\section{Nível 10}

\begin{itemize}
\item \url{}
\item \url{}
\end{itemize}



 
% Chapter Template

\chapter{Conclusão} % Main chapter title

\label{Chapter5} % Change X to a consecutive number; for referencing this chapter elsewhere, use \ref{ChapterX}

%----------------------------------------------------------------------------------------
%	SECTION 1
%----------------------------------------------------------------------------------------

Esse trabalho vem preencher um pouco da falta de um bom material didático sobre Teoria do Número aplicada em Competições de Programação.

Pelo fato de ser bem modular, é possível incrementar o conteúdo do mesmo com problemas e teorias de maneira coerente.

Assim, espero que esse trabalho ajude outros alunos da mesma forma que teria me ajudado no começo dos meus estudos. 
 

%----------------------------------------------------------------------------------------
%	THESIS CONTENT - APPENDICES
%----------------------------------------------------------------------------------------

\appendix % Cue to tell LaTeX that the following "chapters" are Appendices

% Include the appendices of the thesis as separate files from the Appendices folder
% Uncomment the lines as you write the Appendices

% Appendix A

\chapter{Curiosidades da ACM-ICPC} % Main appendix title

\label{AppendixA} % For referencing this appendix elsewhere, use \ref{AppendixA}

ACM-ICPC (International Collegiate Programming Contest) é uma competição de programação
de várias etapas e baseada em equipe. O principal objetivo é encontrar algoritmos
eficientes, que resolvem os problemas abordados pela competição, o mais rápido
possível.


Nos últimos anos a ACM-ICPC teve um crescimento significativo. Se compararmos
o número de competidores, temos que de 1997 (ano em que começou o patrocinio
da IBM) até 2014 houve um aumento maior que $1500\%$, totalizando 38160
competidores de 2534 universidades em 101 países ao redor do mundo.

Para mais informa\c{c}\~oes sobre as competi\c{c}\~oes passadas acesse \href{icpc.baylor.edu}{icpc.baylor.edu}.

\begin{figure}[p]
    \centering
    \includegraphics[width=1\linewidth]{Figures/grafico.png}
    \caption{Crescimento do n\'umero de participantes por ano.}
\end{figure}


% Appendix Template

\chapter{Juízes Online (Online Judges)} % Main appendix title

\label{AppendixB} % Change X to a consecutive letter; for referencing this appendix elsewhere, use \ref{AppendixX}

Online Judges são plataformas online que contam com um banco de dados com diversos tipos de problemas de competições de programação, e com um sistema de correção online.

Para comprovar que seu programa resolve o problema dado, basta enviar o código fonte da sua solução (em geral escrito em C++ ou JAVA) para uma dessas plataformas.

Alguns desses Online Judges são citados em seguida.


\section{Ahmed-Aly}

Juíz online que permite criar e participar de competições de programação online com problemas de outros \textit{Online Judges}.

Também possibilita o gerenciamento das suas competições de programação e de seus amigos.

Para mais problemas de \textit{Teoria dos Números} acesse \href{http://a2oj.com/Category.jsp?ID=41}{http://a2oj.com/Category.jsp?ID=41}.

Site: \href{http://a2oj.com/}{http://a2oj.com/}


\section{UVa} 
	
Criado em 1995 pelo matemático Miguel Ángel Revilla, é atualmente um dos Online Judges mais famoso entre os participantes da ACM-ICPC.

É hospedado pela \href{http://www.uva.es/export/sites/uva/}{Universidade de Valhadolide} e conta com mais de 100000 usuários registrados.

Site: \href{https://uva.onlinejudge.org/}{https://uva.onlinejudge.org/}


\section{URI}

Projeto desenvolvido pelo Departamento de Ciência da Computação da \href{http://www.uri.br/}{Universidade Regional Integrada}.
Contando com um enorme repositório com problemas de competições de programação, o principal objetivo desse projeto é proporcionar uma 
plataforma para a prática de programação e compartilhamento de conhecimentos.

Site: \href{https://www.urionlinejudge.com.br/}{https://www.urionlinejudge.com.br/}

\section{Topcoder} 

Empresa que administra competições de programação nas linguagens Java, C++ e C$\#$.

É responsável também por aplicar competições de design e desenvolvimento de software.

Site: \href{https://www.topcoder.com/}{https://www.topcoder.com/}

\section{Codeforces}

Site Russo dedicado às competições de programação. 

Em 2013, Codeforces superou Topcoder com relação ao número de usuários ativos, apesar de ter sido criado quase 10 anos depois.

O estilo de problemas que esse site aplica é similar aos problemas encontrados na ACM-ICPC.

Site: \href{http://codeforces.com/}{http://codeforces.com/}

\section{CodeChef}

Iniciativa educacional sem fins lucrativos lançada em 2009 pela \href{http://www.directi.com/}{Direct}.

É uma plataforma de progamação competitiva que suporta mais de 35 linguagens de programação.

Site: \href{https://www.codechef.com/}{https://www.codechef.com/}



%\input{Appendices/AppendixC}

%----------------------------------------------------------------------------------------
%	BIBLIOGRAPHY
%----------------------------------------------------------------------------------------

\printbibliography[heading=bibintoc]

%----------------------------------------------------------------------------------------

\end{document}  

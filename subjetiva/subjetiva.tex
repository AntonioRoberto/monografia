%%%%%%%%%%%%%%%%%%%%%%%%%%%%%%%%%%%%%%%%%
% Masters/Doctoral Thesis 
% LaTeX Template
% Version 2.1 (2/9/15)
%
% This template has been downloaded from:
% http://www.LaTeXTemplates.com
%
% Version 2.0 major modifications by:
% Vel (vel@latextemplates.com)
%
% Original authors:
% Steven Gunn  (http://users.ecs.soton.ac.uk/srg/softwaretools/document/templates/)
% Sunil Patel (http://www.sunilpatel.co.uk/thesis-template/)
%
% License:
% CC BY-NC-SA 3.0 (http://creativecommons.org/licenses/by-nc-sa/3.0/)
%
%%%%%%%%%%%%%%%%%%%%%%%%%%%%%%%%%%%%%%%%%

%----------------------------------------------------------------------------------------
%	PACKAGES AND OTHER DOCUMENT CONFIGURATIONS
%----------------------------------------------------------------------------------------

\documentclass[
11pt, % The default document font size, options: 10pt, 11pt, 12pt
%oneside, % Two side (alternating margins) for binding by default, uncomment to switch to one side
brazilian, % ngerman for German
singlespacing, % Single line spacing, alternatives: onehalfspacing or doublespacing
%draft, % Uncomment to enable draft mode (no pictures, no links, overfull hboxes indicated)
%nolistspacing, % If the document is onehalfspacing or doublespacing, uncomment this to set spacing in lists to single
%liststotoc, % Uncomment to add the list of figures/tables/etc to the table of contents
%toctotoc, % Uncomment to add the main table of contents to the table of contents
%parskip, % Uncomment to add space between paragraphs
]{MastersDoctoralThesis} % The class file specifying the document structure

\usepackage[utf8]{inputenc} % Required for inputting international characters
\usepackage[T1]{fontenc} % Output font encoding for international characters

\usepackage{palatino} % Use the Palatino font by default

\usepackage[backend=bibtex,style=authoryear,natbib=true]{biblatex} % User the bibtex backend with the authoryear citation style (which resembles APA)

\addbibresource{example.bib} % The filename of the bibliography

\usepackage[autostyle=true]{csquotes} % Required to generate language-dependent quotes in the bibliography
\usepackage{amssymb}
\usepackage{amsmath}
\usepackage{hyperref}
\usepackage{algorithm}
\usepackage[noend]{algpseudocode}
\usepackage{hyperref}
%\usepackage[]{algorithm2e}
%\usepackage{algorithmic}
%\usepackage[colorlinks=true,linkcolor=blue]{hyperref} 

\newtheorem{theorem}{Teorema}
\newtheorem{definition}{Definição}
\newtheorem{proof}{Demonstracão}
\newtheorem{corollary}{Corolário}

\makeatletter
\def\BState{\State\hskip-\ALG@thistlm}
\makeatother


%----------------------------------------------------------------------------------------
%	THESIS INFORMATION
%----------------------------------------------------------------------------------------

\thesistitle{Teoria dos Números e Computação: Uma abordagem utilizando problemas de competições de programação} % Your thesis title, this is used in the title and abstract, print it elsewhere with \ttitle
\supervisor{Dr. Carlos Eduardo Ferreira} % Your supervisor's name, this is used in the title page, print it elsewhere with \supname
\examiner{} % Your examiner's name, this is not currently used anywhere in the template, print it elsewhere with \examname
\degree{Bacharel em Ciência da Computação} % Your degree name, this is used in the title page and abstract, print it elsewhere with \degreename
\author{Antonio R. de Campos Junior} % Your name, this is used in the title page and abstract, print it elsewhere with \authorname
\addresses{} % Your address, this is not currently used anywhere in the template, print it elsewhere with \addressname

\subject{Ciência da Computação} % Your subject area, this is not currently used anywhere in the template, print it elsewhere with \subjectname
\keywords{} % Keywords for your thesis, this is not currently used anywhere in the template, print it elsewhere with \keywordnames
\university{\href{http://www5.usp.br/}{Universidade de São Paulo}} % Your university's name and URL, this is used in the title page and abstract, print it elsewhere with \univname
\department{\href{http://www.ime.usp.br/}{Instituto de Matemática e Estatística}} % Your department's name and URL, this is used in the title page and abstract, print it elsewhere with \deptname
\group{\href{http://researchgroup.university.com}{Research Group Name}} % Your research group's name and URL, this is used in the title page, print it elsewhere with \groupname
\faculty{\href{http://faculty.university.com}{Faculty Name}} % Your faculty's name and URL, this is used in the title page and abstract, print it elsewhere with \facname

\hypersetup{pdftitle=\ttitle} % Set the PDF's title to your title
\hypersetup{pdfauthor=\authorname} % Set the PDF's author to your name
\hypersetup{pdfkeywords=\keywordnames} % Set the PDF's keywords to your keywords

\begin{document}

\frontmatter % Use roman page numbering style (i, ii, iii, iv...) for the pre-content pages

\pagestyle{plain} % Default to the plain heading style until the thesis style is called for the body content

%----------------------------------------------------------------------------------------
%	TITLE PAGE
%----------------------------------------------------------------------------------------

\begin{titlepage}
\begin{center}

\textsc{\LARGE \univname}\\[1.5cm] % University name
\textsc{\Large Trabalho de Formatura \\ Parte Subjetiva}\\[0.5cm] % Thesis type

\HRule \\[0.4cm] % Horizontal line
{\huge \bfseries \ttitle}\\[0.4cm] % Thesis title
\HRule \\[1.5cm] % Horizontal line
 
\begin{minipage}{0.4\textwidth}
\begin{flushleft} \large
\emph{Autor:}\\
\href{http://www.ime.usp.br/~arcjr}{\authorname} % Author name - remove the \href bracket to remove the link
\end{flushleft}
\end{minipage}
\begin{minipage}{0.4\textwidth}
\begin{flushright} \large
\emph{Supervisor:} \\
\href{http://www.ime.usp.br/~cef/}{\supname} % Supervisor name - remove the \href bracket to remove the link  
\end{flushright}
\end{minipage}\\[3cm]
\leavevmode \\
\leavevmode \\
\leavevmode \\
\leavevmode \\

\large \textit{Tese apresentada em cumprimento dos requisitos para o curso \\ \degreename}\\[0.3cm] % University requirement text
%\textit{in the}\\[0.4cm]
\deptname\\[2cm] % Research group name and department name
 
{\large \today}\\[4cm] % Date
%\includegraphics{Logo} % University/department logo - uncomment to place it
 
\vfill
\end{center}
\end{titlepage}

%----------------------------------------------------------------------------------------
\newpage
%----------------------------------------------------------------------------------------

\chapter*{Parte Subjetiva}
\thispagestyle{empty}

\section {Introdução}

Desde que eu era calouro eu me interessei por competições de programação. Logo no segundo ano, em 2011, alguns amigos e eu começamos a estudar praticamente todos os dias, com a esperança de nos classificarmos para final brasileira da maratona de programação pelo menos uma vez antes de terminar a graduação.

Hoje em dia, conquistamos alguns resultados expressivos, como:

\begin{itemize}
\item 1ª Colocação na final brasileira da maratona de programação em 2014;
\item 2ª Colocação na final brasileira da maratona de programação em 2013;
\item Classificados para final mundial da maratona de programação em 2014 e 2015;
\end{itemize}

A experiência que eu adquiri nos últimos anos me preparando para ACM-ICPC (International Collegiate Programming Contest), meu interesse sobre Matemática e a falta de um bom material que unifique diversos tópicos relacionados à Teoria dos Números e Programação, me levaram a trabalhar nesse tema.



\section {Desafios}
O maior desafio para mim, foi organizar as idéias.
Eu já tinha muita experiência em resolver problemas relacionados à Teoria dos Números, porém o desafio foi fazer com que esse trabalho tivesse um fluxo de conteúdo que fosse fácil de acompanhar e ao mesmo tempo agregasse valor aos leitores.



\section {Parecer Pessoal}
Eu achei muito interessante trabalhar nesse tema. Esse material teria definitivamente me ajudado na preparação para ACM-ICPC quando eu comecei meus estudos.

Espero que esse trabalho possa ajudar outras pessoas, que como eu, adoram competições de programação.


\section {Próximos Passos}
Já não posso mais participar da ACM-ICPC, pelo fato de eu ter atendido à 2 finais mundiais.
Assim, o que me resta é continuar participando de competições de programação online, como codeforces e topcoder.

Pretendo manter minha presença nessas competições e melhorar minhas habilidades cada vez mais. 
\clearpage


\end{document}  
